\documentclass[12pt,letterpaper]{article}
\usepackage[utf8]{inputenc}
\usepackage{amsmath,amsthm,amsfonts,amssymb,amscd}
\usepackage[table]{xcolor}
\usepackage[margin=4cm]{geometry}
\usepackage{ragged2e}
\usepackage{graphicx}
\usepackage{multicol}
\usepackage{csquotes}
\usepackage{listings}
\usepackage{float}

\usepackage{color}
%\usepackage[brazil]{babel}

\newlength{\tabcont}
\setlength{\parindent}{0.3in}
\setlength{\parskip}{0.05in}

\definecolor{dkgreen}{rgb}{0,0.6,0}
\definecolor{gray}{rgb}{0.5,0.5,0.5}
\definecolor{mauve}{rgb}{0.58,0,0.82}

\lstset{frame=tb,
  language=C,
  aboveskip=3mm,
  belowskip=3mm,
  showstringspaces=false,
  columns=flexible,
  basicstyle={\small\ttfamily},
  numbers=none,
  numberstyle=\tiny\color{gray},
  keywordstyle=\color{blue},
  commentstyle=\color{dkgreen},
  stringstyle=\color{mauve},
  breaklines=true,
  breakatwhitespace=true,
  tabsize=2
}

\title{MAC0328 - Lista 1}
\author{Lucas Santos}
\date{Agosto 2017}

\begin{document}

\maketitle

\begin{enumerate}
    \item Primeiro é importante notar que um grafo com $V$ vértices possui $V * (V - 1)$ arcos diferentes. Para saber quantos grafos diferentes podemos formar com $V$ vértices e $A$ arcos, basta tomar a combinação de $V * (V - 1)$ arcos tomados $A$ a $A$.\\
    $$C = \frac{n!}{p! * (n-p)!}$$\\
    $$C = \frac{(V * (V - 1))!}{A! * (V * (V - 1) - A)!}$$
    
    \item Cada arco contribui com 1 na soma de graus de saída (e de entrada, também).
    
    \item 
    \subitem (a) Para grafos usando matriz de adjacência (complexidade do tempo de execução $\theta(V^2)$):
    \begin{lstlisting}
        int GRAPHoriented (Graph G) {
            Vertex v, w;
            for (v = 0; v < G->V; v++)
                for (w = v + 1; w < G->V; w++)
                    if (G->adj[v][w] != G->adj[w][v])
                        return 0;
            return 1;
        }
    \end{lstlisting}
    
    \subitem (b) Para grafos usando listas de adjacência (complexidade do tempo de execução $O(V+A^2)$):
    \begin{lstlisting}
        int GRAPHoriented (Graph G) {
            Vertex v;
            link p, q;
            for (v = 0; v < G->V; v++) {
                for (p = G->adj[v]; p != NULL; p = p->next) {
                    for (q = G->adj[p->w]; q != NULL && q->w != v; q = q->next);
                    if (q == NULL)
                        return 1;
                }
            }
            return 0;
        }
    \end{lstlisting}
    
    \item Assumindo $G$ um grafo não-dirigido, então o grau de um vértice é o número de arestas no seu leque. Segue abaixo uma implementação em C.
    \subitem (a) Para grafos usando matriz de adjacência (complexidade do tempo de execução $\theta(V)$):
    \begin{lstlisting}
        int GRAPHdeg (Graph G, int v) {
            Vertex w;
            int deg = 0;
            for (w = 0; v < G->V; w++)
                if (G->adj[v][w] == 1)
                    deg++;
            return deg;
        }
    \end{lstlisting}
    
    \subitem (b) Para grafos usando listas de adjacência (complexidade do tempo de execução $O(V)$):
    \begin{lstlisting}
        int GRAPHdeg (Graph G, int v) {
            link p;
            int deg = 0;
            for (p = G->adj[v]; p != NULL; deg++, p = p->next);
            return deg;
        }
    \end{lstlisting}
    
    \item
    \subitem (a) Para grafos usando matriz de adjacência (complexidade do tempo de execução $\theta(V)$):
    \begin{lstlisting}
        int GRAPHisolated (Graph G, int v) {
            Vertex w;
            for (w = 0; w < G->V; w++)
                if (G->adj[v][w] == 1 || G->adj[w][v] == 1)
                    return 0;
            return 1;
        }
    \end{lstlisting}
    
    \subitem (b) Para grafos usando listas de adjacência (complexidade do tempo de execução $O(V + A)$):
    \begin{lstlisting}
        int GRAPHisolated (Graph G, int v) {
            Vertex w;
            link p;
            if (G->adj[v] != NULL)
                return 0;
            for (w = 0; w < G->V; w++)
                for (p = G->adj[w]; p != NULL; p = p->next)
                    if (p->w == v)
                        return 0;
            return 1;
        }
    \end{lstlisting}
    
    \item
    \subitem (a) Para grafos usando matriz de adjacência (complexidade do tempo de execução $\theta(1)$):
    \begin{lstlisting}
        int GRAPHadj (Graph G, int v, int w) {
            return G->adj[w][v];
        }
    \end{lstlisting}
    
    \clearpage
    
    \subitem (b) Para grafos usando listas de adjacência (complexidade do tempo de execução $O(V)$):
    \begin{lstlisting}
        int GRAPHadj (Graph G, int v, int w) {
            Vertex w;
            link p;
            for (p = G->adj[w]; p != NULL; p = p->next)
                if (p->w == v)
                    return 1;
            return 0;
        }
    \end{lstlisting}
    
    \item O código é o mesmo, independente se a estrutura de dados usadas para o grafo é matriz de adjacência ou listas de adjacência. A complexidade do tempo de execução, entretanto, é diferente. O primeiro é $\theta(V^2)$ enquanto o segundo é $O(V^3)$ e $\Omega(V^2)$. Note que a diferença na complexidade ocorre as diferenças de implementação da função que insere o arco. Segue o código em C:
    \begin{lstlisting}
        Graph GRAPHknightRand (int n) {
            Graph G = GRAPHinit (n * n);
            int x[8] = { -1, 1, 2, 2, 1, -1, -2, -2 };
            int y[8] = { 2, 2, 1, -1, -2, -2, -1, 1 };
            Vertex l, c;
            Vertex v;
            int i;
            for (v = 0; v < n * n; v++) {
                l = v / n;
                c = v % n;
                for (i = 0; i < 8; i++)
                    GRAPHinsertArc (G, v, (l + x[i]) * n + (c + y[i]));
            }
            return G;
        }
    \end{lstlisting}
    
    \item Segue abaixo uma tabela com os valores de $V$ e $A$ assim como o tempo que foi gasto para construir o grafo $G$ com essas configuranções utilizando o construtor aleatório 1.
    
    \begin{table}[H]
    \centering
    \caption{exercício 8}
    \label{my-label}
    \begin{tabular}{rrc}
    \hline
    \multicolumn{1}{c}{\textbf{V}} & \multicolumn{1}{c}{\textbf{A}} & \textbf{Tempo gasto (em segundos)} \\ \hline
    10.000                         & 10.000                         & 0.48                               \\ \hline
    10.000                         & 10.000                         & 0.49                               \\ \hline
    10.000                         & 1.000.000                      & 0.62                               \\ \hline
    10.000                         & 10.000.000                     & 1.66                               \\ \hline
    10.000                         & 50.000.000                     & 8.56                               \\ \hline
    10.000                         & 95.000.000                     & 34.36                              \\ \hline
    10.000                         & 99.990.000                     & Não finalizou                                   \\ \hline
    \end{tabular}
    \end{table}
    
    \item Abaixo, dois histogramas com valores diferentes de $V$ e $A$. Os valores eram o que eu esparava, isto é, bem distribuido e com alguns poucos casos destoantes.
    
    \begin{figure}[H]
    \caption{$V = 100$, $A = 9000$, média $= 49.5$, variância $= 28.86$}
    \centering
    \includegraphics[width=0.75\textwidth]{Figure_1.png}
    \end{figure}
    
    \begin{figure}[H]
    \caption{$V = 1000$, $A = 5000$, média $= 499.5$, variância $= 288.67$}
    \centering
    \includegraphics[width=0.75\textwidth]{Figure_2.png}
    \end{figure}
    
    \item Uma regra que é possível gerar vários grafos é uma que mistura os construtores do rei no xadrez e o de arcos randômicos (tipo 2). A ideia é inserir os arcos do rei no xadrez com uma probabilidade $p$ \textit{(abaixo definida como 0.5)}. A implementação em C é dada por:
    \begin{lstlisting}
        Graph GRAPHbuildChessKingRand (int n) {
            Graph G = GRAPHinit (n * n);
            Vertex l, c, i, j;
            Vertex v;
            double p = 0.5;
            for (v = 0; v < n * n; v++) {
                l = v / n; // Linha do vértice atual
                c = v % n; // Coluna do vértice atual
                for (i = l - 1; i <= l + 1; i++)
                    for (j = c - 1; j <= c + 1; j++)
                        if (i >= 0 && i < n && j >= 0 && j < n && rand() / (RAND_MAX + 1.0) < p)
                            GRAPHinsertArc (G, v, i * n + j);
            }
            return G;
        }
    \end{lstlisting}
    
    Esse método constroi um grafo de forma aleatório e isso implica que executando esse método duas vezes não é garantido a construção do mesmo grafo. O grafo $\pi$ e o grafo da Monalisa não são aleatórios, uma vez que executando o método diversas vezes com a mesma entrada, é garantido que o mesmo grafo será construido.
    
    Realizando alguns experimentos com a regra sugerida (rei no xadrez com probabilidade na existência das arestas), foram feitos 5 experimentos com $p = 0.5$ e $n \in \{ 5, 10, 20, 50, 100 \}$ e foi avaliado algumas propriedades do grafo. São elas:
    
    \subitem (a) Grau de entrada e saída: sempre menor ou igual a 6 com uma média próxima de 2. Acredito que tenha sido próximo de 2, e não de 3 (o esperado), devido aos nós nos bordos e quinas.
    \subitem (b) Embora possível, é bem raro a ocorrência de vértices isolados, fontes ou sorvedouros. Nos casos que foram testados, não houveram nenhuma ocorrência.
    \subitem (c) Novamente, embora possível, não ocorreram casos onde o grafo é completo ou um torneio.
    
\end{enumerate}
\end{document}
